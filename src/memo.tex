\expandafter\ifx\csname ifdraft\endcsname\relax

\documentclass{jsarticle}
\usepackage{amsmath} % alignに必要

% コメント用
\newcommand{\comment}[1]{\qquad &\text{$(#1)$}}


\begin{document}
\fi

\section{剰余の定理}

$P(a)をaに関する多項式とする$
\begin{align*}
   P(a) &= (a - k)Q(a) + R \comment{P(a)を(a-k)で割る} \\
   ここで、a = kを代入すると \\
   P(k) &= (k - k)Q(k) + R \\
   P(k) &= R = 0 \\
   よって P(a) &= (a-k)Q(a) + P(k)  \\
   つまり P(k) &= 0 \Leftrightarrow P(a)を(a-k)で割り切れる。
\end{align*}

具体例

$P(a) = a^n - b^n$とすると、$P(b) = b^n - b^n = 0$なので、$(a-b)はa^n - b^nを割り切れる$

実際に、$a^n - b^n = (a - b)(\sum_{k = 0}^{n-1} a^{n-1-k}b^k)$

$n = 3のとき、a^3 - b^3 = (a-b)(a^2 + ab + b^2)$



\section{$-1 \times -1 = 1の証明$}

\begin{align*}
    -1 + 1 &= 0                           \comment{マイナスの定義 -a + a = 0} \\
    -1 \times ( -1 + 1) &= -1 \times 0    \comment{-1をかける}\\
    (-1) \times (-1) + (-1) \times 1 &= 0 \comment{分配法則と、a \times 0 = 0} \\
    (-1) \times (-1) + (-1) &= 0          \comment{a \times 1 = a} \\
    (-1) \times (-1) &= 1                 \comment{両辺に1を足す}
\end{align*}



\expandafter\ifx\csname ifdraft\endcsname\relax
\end{document}
\fi
