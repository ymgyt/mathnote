\expandafter\ifx\csname ifdraft\endcsname\relax

\documentclass{jlreq}
\usepackage{amsmath} % alignに必要

% コメント用
\newcommand{\comment}[1]{\qquad &\text{$(#1)$}}


\begin{document}
\fi
\section{統計}

\subsection{相関係数(correlation coefficient)}

\paragraph{分散}
$n個の資料x_1, x_2, \dots, x_nの分散\sigma_{x}^{2}は$

\begin{gather*}
    \sigma_{x}^2 = \frac{1}{n} \sum (x_i - \bar{x})
\end{gather*}

と定義する。偏差の2乗和。

\paragraph{共分散}

$2変量(x,y)をもつn個の資料(x_1, y_1), (x_2, y_2), \dots (x_n, y_n)における共分散 \sigma_{xy}は$

\begin{gather*}
    \sigma_{xy} = \frac{1}{n} \sum (x_i - \bar{x}) (y_i - \bar{y})\\
    (ただし、\bar{x} = \frac{1}{n} \sum x_i, \bar{y} = \frac{1}{n} \sum y_i)\\
\end{gather*}

と定義する。$xの偏差とyの偏差の積の平均。$

\paragraph{相関係数の定義}

$2変量(x,y)を持つn個の資料を(x_1,y_1), (x_2, y_2), \dots (x_n, y_n)とする。このとき、相関係数rは$

\[r = \frac{\sigma_{xy}}{\sigma_x \sigma_y}\]

と定義される。$xとyの共分散を標準偏差の積で割る。$

$また、分子のnとxとyの標準偏差をかけることで生まれるnが打ち消しあうので以下の定義も可能$

\begin{align*}
    r &= \frac
        {\frac{1}{n} \sum (x_i - \bar{x})(y_i - \bar{y}) }
        { \sqrt{\frac{1}{n} \sum (x_i - \bar{x})^2}  \sqrt{\frac{1}{n} \sum (y_i - \bar{y})^2}} \\
      &= \frac
        {\sum (x_i - \bar{x})(y_i - \bar{y})}
        { \sqrt{\sum (x_i - \bar{x})^2} \sqrt{\sum (y_i - \bar{y})^2}}
\end{align*}

\paragraph{相関係数の絶対値が1以下の理由}

\end{document}
\fi
