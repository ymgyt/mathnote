\expandafter\ifx\csname ifdraft\endcsname\relax

\documentclass{jlreq}
\usepackage{amsmath} % alignに必要

% コメント用
\newcommand{\comment}[1]{\qquad &\text{$(#1)$}}


\begin{document}
\fi
\section{統計}

\subsection{相関係数(correlation coefficient)}

\paragraph{分散}
$n個の資料x_1, x_2, \dots, x_nの分散\sigma_{x}^{2}は$

\begin{gather*}
    \sigma_{x}^2 = \frac{1}{n} \sum (x_i - \bar{x})
\end{gather*}

と定義する。偏差の2乗和。

\paragraph{共分散}

$2変量(x,y)をもつn個の資料(x_1, y_1), (x_2, y_2), \dots (x_n, y_n)における共分散 \sigma_{xy}は$

\begin{gather*}
    \sigma_{xy} = \frac{1}{n} \sum (x_i - \bar{x}) (y_i - \bar{y})\\
    (ただし、\bar{x} = \frac{1}{n} \sum x_i, \bar{y} = \frac{1}{n} \sum y_i)\\
\end{gather*}

と定義する。$xの偏差とyの偏差の積の平均。$

\paragraph{相関係数の定義}

$2変量(x,y)を持つn個の資料を(x_1,y_1), (x_2, y_2), \dots (x_n, y_n)とする。$

このとき、相関係数rは

\[r = \frac{\sigma_{xy}}{\sigma_x \sigma_y}\]

と定義される。$xとyの共分散を標準偏差の積で割る。$

$また、分子のnとxとyの標準偏差をかけることで生まれるnが打ち消しあうので以下の定義も可能$

\begin{align*}
    r &= \frac
        {\frac{1}{n} \sum (x_i - \bar{x})(y_i - \bar{y}) }
        {\sqrt{\frac{1}{n} \sum (x_i - \bar{x})^2} \sqrt{\frac{1}{n} \sum (y_i - \bar{y})^2}} \\
      &= \frac
        {\sum (x_i - \bar{x})(y_i - \bar{y})}
        {\sqrt{\sum (x_i - \bar{x})^2} \sqrt{\sum (y_i - \bar{y})^2}}
\end{align*}

\paragraph{相関係数の絶対値が1以下の証明}

$相関係数rが -1 \leq r \leq 1となることを証明する。$

\begin{align*}
    \intertext{まず}
    x_i - \bar{x} &= a_i &   y_i - \bar{y} &= b_i  & a_i b_i &= c_i \\
    \sum {a_i}^2 &= A    &  \sum {b_i}^2 &= B      & \sum a_i b_i &= C \\
    \intertext{とおくと}
\end{align*}
\begin{align*}
    -1 &\leq r \leq 1 \\
    -1 &\leq \frac {C}{\sqrt{A} \sqrt{B}} \leq 1 \\
    -c &\leq x \leq c \leftrightarrow x^2 \leq c^2 なので \\
    &(\frac{C}{\sqrt{A}\sqrt{B}})^2 \leq 1^2 \\
    &\frac{C^2}{AB} \leq 1 \\
    &C^2 \leq AB \qquad(AB \geq 0) \\
    &C^2 - AB \leq 0 \\
    &AB - C^2 \geq 0
    \intertext{のように変形できるのでこれを証明する}
    \intertext{ここで以下の関数を考える}
    f(x) &= \sum (a_i x - b_i)^2 \\
    &= \sum ({a_i}^2 x^2 -2a_i b_i x + {b_i}^2) \\
    &= \sum {a_i}^2 x^2 -2(\sum a_i b_i) + \sum {b_i}^2 \\
    &= Ax^2 -2Cx + B
    \intertext{これを平方完成すると}
    &= A(x^2 - \frac{2C}{A}x) + B \\
    &= A\{ (x - \frac{C}{A})^2 \} - \frac{C^2}{A} + B \\
    &= A\{ (x - \frac{C}{A})^2 \} + \frac{AB - C^2}{A} \\
    \intertext{f(x)は2乗の和なので、f(x) \geq 0}
    \intertext{$x = \dfrac{C}{A}のときf(x)は最小値 \dfrac{AB - C^2}{A}をとるので$}
    \frac{AB - C^2}{A} &\geq 0 \\
    AB - C^2 &\geq 0 \qquad (A \geq 0)
    \intertext{となり、証明できた。}
\end{align*}

\end{document}
\fi
